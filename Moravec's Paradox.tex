\documentclass{article}
\usepackage[utf8]{inputenc}

\title{Moravec's Paradox}
\author{Sahitya Charchit }
\date{27 July 2021}

\begin{document}

%-------------------------------
    %	TITLE SECTION
    %-------------------------------
    
    \fancyhead{}
    \hrule \medskip % Upper rule
    \begin{minipage}{0.295\textwidth} 
    \raggedright
    \footnotesize
    Sahitya Charchit\hfill\\   
    19111050\hfill\\
    BIOMED 5th SEM
    \end{minipage}
    \begin{minipage}{0.4\textwidth} 
    \centering 
    \normalsize
    ASSIGNMENT 3\\
    \Large 
    Moravec's Paradox\\ 
    \end{minipage}
    \begin{minipage}{0.295\textwidth} 
    \raggedleft
    \today\hfill\\
    \end{minipage}
    \medskip\hrule 
    \bigskip
\section{Summary}
The Moravec Paradox is related to the findings of artificial intelligence and robotics researchers. Contrary to popular belief, thinking requires very little computing power, while sensory motor skills require a lot of computing resources.  In the 1980s, Hans Moravec, Rodney Brooks, Marvin Minsky, and others defined this idea. In 1988, Moravec said that it is easy for computers to perform adult-level IQ tests or checks, but it is difficult or impossible to give them a year of cognitive and motor skills.  Minsky also emphasized that unconscious human abilities are the most difficult to transform. "In general, we don't know much about what our brains are best at," he wrote, adding, "We understand simple processes that perform poorly better than more complex processes that run perfectly."  Moravec It is suggested to use evolution as a possible solution to the problem. Dilemma. All human abilities are realized physiologically through equipment created by natural selection. Natural selection tends to retain design improvements and optimizations throughout the development process. Natural selection has more room for productivity improvement. Cultivate talents with age. Since abstract cognition is a relatively new concept, it should not be expected to be implemented efficiently.  Examples of skills developed over millions of years: facial recognition, spatial movement, assessing people’s motivation, catching the ball, voice recognition, setting appropriate goals, focusing on interesting things; anything related to perception, attention, visualization, movement Skills, social skills and other related things.  Some examples of new skills—mathematics, engineering, games, logic, and scientific reasoning—are difficult for us because they are not what our bodies and brains were originally designed for. Recently acquired, in the historical period, should not be improved after thousands of years, mainly the result of cultural evolution.   All in all, the difficulty of reverse engineering human talent should be roughly proportional to the development time of the skill in animals. Since the early human talents were mostly unconscious, it seemed easy for us. We should assume that seemingly simple skills are difficult to reconstruct, but skills that require hard work may not be difficult to develop.
\end{document}
\documentclass{article}
\usepackage[utf8]{inputenc}

\title{Godel's Incompleteness Theorem}
\author{Sahitya Charchit }
\date{21 July 2021}

\begin{document}

%-------------------------------
    %	TITLE SECTION
    %-------------------------------
    
    \fancyhead{}
    \hrule \medskip % Upper rule
    \begin{minipage}{0.295\textwidth} 
    \raggedright
    \footnotesize
    Sahitya Charchit\hfill\\   
    19111050\hfill\\
    BIOMED 5th SEM
    \end{minipage}
    \begin{minipage}{0.4\textwidth} 
    \centering 
    \large 
    ASSIGNMENT 2\\
    \normalsize 
    Godel's Incompleteness Theorem\\ 
    \end{minipage}
    \begin{minipage}{0.295\textwidth} 
    \raggedleft
    \today\hfill\\
    \end{minipage}
    \medskip\hrule 
    \bigskip
\section{Summary}
Gödel's incompleteness theorems are two propositions of mathematical logic, involving the provable limits of formal axioms. These results were published by Kurt Gödel in 1931 and are important for mathematical logic and mathematical philosophy.  These theorems are widely but not universally interpreted to indicate that  Hilbert's  program used to find a complete and consistent set of axioms is impossible for all mathematics. The first theorem of incompleteness says that there is no consistent axiom system , and its theorems can be enumerated through effective methods (that is, algorithm ), which can prove all the truths about natural number arithmetic.  For every such consistent formal system, there will always be statements about  natural numbers that are correct, but they cannot be proven within the framework of the system. The second  incomplete set is an extension of the first, indicating that the system  cannot prove its consistency.  Mathematics tries to prove the truth of the statement on the basis of these axioms and definitions, but sometimes the axioms are not enough. Sometimes, axiom  leads to a paradox that requires a new set of axioms. Sometimes  axioms are not enough. This is not enough, so a new axiom may be needed to prove the desired result .  Gödel’s incompleteness theorem shows that, in fact, every logical system  has contradictions or statements that cannot be proved. Gödel tried to answer  questions: "Can I prove that mathematics is consistent?" and  "If I have a true statement, can I prove that it is true?"  According to Gödel's completeness theorem, it has one The proposition of the set of axioms  is provable if and only if the proposition is true for each model of the set of axioms.  The axiom as a proposition is false. But if the consistency of a set of axioms  is unprovable, it means that their  axioms must have a model in which the consistency statement is wrong.
\end{document}
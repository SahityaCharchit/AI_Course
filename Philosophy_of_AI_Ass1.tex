\documentclass{article}
\usepackage[utf8]{inputenc}

\title{AI_PHILOSOPHY}
\author{Sahitya Charchit }
\date{July 2021}

\begin{document}

%-------------------------------
    %	TITLE SECTION
    %-------------------------------
    
    \fancyhead{}
    \hrule \medskip % Upper rule
    \begin{minipage}{0.295\textwidth} 
    \raggedright
    \footnotesize
    Sahitya Charchit\hfill\\   
    19111050\hfill\\
    BIOMED 5th SEM
    \end{minipage}
    \begin{minipage}{0.4\textwidth} 
    \centering 
    \large 
    ASSIGNMENT 1\\
    \normalsize 
    Philosophy of Artificial Intelligence\\ 
    \end{minipage}
    \begin{minipage}{0.295\textwidth} 
    \raggedleft
    \today\hfill\\
    \end{minipage}
    \medskip\hrule 
    \bigskip

\section{Introduction}
This article aims to understand the entire AI journey from an assumption to its present state, where it aims to go to discuss with its pros and cons that affect the whole universe. The philosophy of artificial intelligence has its roots in meeting issues as follows:
\begin{itemize}
\item Can a machine ever be seen as smart? If yes, would it solve an issue of using the same protocols as a human?
\item Can we develop an algorithm that works like the human mind ? If so, can the human brain be called a  computer (with a lot of computing power and network)?
\item If we can closely imitate human thinking, will machines have the same subjective mental state as humans, their own thinking, and the perception of conscious experience?  is truly conscious. It is not an algorithm designed to not feel or react to anything.
\end{itemize}
Questions of this nature have attracted the scientific community in all fields, from mathematicians, physicists, neuroscientists to entrepreneurs who want to use this powerful but unpredictable  technology to improve the quality of life.
\section{A machine with intellect, consciousness, and a state of mind}
This philosophical query brings at the side of the ’difficult problem of awareness’ and refers a destiny with ’strong AI’. A sturdy AI apart from being intelligent could additionally possess the essence of experiencing qualia. it would now not simply realize, however have the feeling of knowing as nicely. we're nevertheless on the very initial level in this realm, and a tremendous research is required here to recognize how would possibly a organic system increase cognizance, and if so, how can it's proved theoretically that it’s displaying the sort of section. diverse theories had been given been scientists which purpose to take diverse processes from quantum mechanics at play to quantifying consciousness into phi, none of them have yet crossed the hypothetical level and given treasured evidence.Scientists like Searle and Lebiniz have hypothesized numerous idea experiments deducing that it would in no way be feasible for gadget to understand what’s it processing, a electricity which handiest aware beings possess. however,there are numerous individuals who beg to vary. it's far a very exciting field to explore and brainstorm (because as of now, simplest beings with brains are asking this question).
\section{
Artificial intelligence covers a wide range of topics: }
\begin{enumerate}
	\item \textbf{Intelligent Agent: }. Intelligent agent is a program that can make decisions or provide services based on the environment, personal information, and past experience. Data is  periodically, pre-programmed or when the user requests it in real time. 

	\item \textbf{Problem Solving: } This is a part of artificial intelligence, which includes more than  methodssuch as trees and heuristic algorithms to solve problems. We can also say that the problem solving agent  is an agent that is result-oriented and always goal-oriented when achieving goals. 

	\item \textbf{Knowledge}:This is information about a domain that can be used to solve problems in that domain. As part of designing a problem-solving program, we need to define how knowledge is presented. The representation scheme is a form of knowledge used in the agent. 

	\item \textbf{Reasoning: }Reasoning is the psychological process of drawing logical conclusions and making predictions based on available  knowledge, facts and beliefs. 

	\item \textbf{Planning:}These are decision-making tasks performed by  robots or computer programs to achieve specific goals. The plan execution is to select a series of actions with a high probability to perform a specific task. 

	\item \textbf{Unsafe knowledge}: When the available knowledge has multiple reasons, causes multiple effects, or the knowledge of causality is incomplete 
	
	\item \textbf{Machine Learning: }This is to obtain knowledge about the ai program by observing its improved environment process and from previous mistakes.

	\item \textbf{Communicating:}Natural language processing includes technologies such as  machine translation into human language, Siri and other voice communication systems, algorithms that can create and publish messages, and social robots, all of which are designed to communicate with processing.

	\item \textbf{Perception:} In artificial intelligence, it is the process of explaining sight, hearing, smell and touch. perception is the process of interpreting, activating, selecting and organizing sensory information from the physical world to perform human behaviour.

	\item \textbf{Performance:} Refers to the actions taken by the algorithm in the real world and is based on the processing data that has been passed to the algorithm.\end{enumerate}
	
\end{document}
